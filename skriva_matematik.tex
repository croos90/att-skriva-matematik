\documentclass[titlepage]{article}


\usepackage[swedish]{babel}
\usepackage{hyperref}
\usepackage{amsmath, amssymb, mathtools}
\usepackage{enumitem}
\usepackage{blindtext}
\usepackage{tcolorbox}
\usepackage{graphicx}

% \definecolor{green}{cmyk}{0.37, 0, 0.94, 0.21, 0.20}

\bibliographystyle{plainurl}

\title{Att skriva matematik \\ \normalsize Förberedande Kurs i matematik}
\author{}
\date{}



\begin{document}

%Titelsida
\begin{titlepage}
    \maketitle
\end{titlepage}

% TOC
\tableofcontents
\thispagestyle{empty}


%%%%%%%%%%%%%%%%%
%%  Inledning  %%
%%%%%%%%%%%%%%%%%

\newpage
\setcounter{page}{1}

\section{Inledning}

På universitet och högskolor är kraven på skriftliga presentationer höga, och det är därmed viktigt att lära sig skriva matematisk text. I detta kompendium ger vi dig tips och råd på hur du skriver matematik på ett snyggt och tydligt sätt.

Vi lyfter saker som är extra viktiga att tänka på vid skrivandet, och använder exempel för att tydliggöra skillnaden mellan bra och mindre bra matematisk text. För den som tänker fortsätta läsa matematik och är intresserad av att läsa mer om matematiskt skrivande rekommenderas \textit{Mathematical writing} av Franco Vivaldi\cite{vivaldi}, som står för inspirationen bakom flera punkter i denna text.


%%%%%%%%%%%%%%%%%%%%%%%%%%%%%%%
%%  Allmänna skrivtips       %%
%%    - Siffor och symboler  %%
%%%%%%%%%%%%%%%%%%%%%%%%%%%%%%%

\section{Allmänna skrivtips}

Börja alltid med att skissa på en lösning på problemet separat innan du börjar skriva själva texten. När du väl har en lösning kan du börja fundera på hur du bäst kan presentera den i text. Det finns många sätt att lösa ett problem på, och ännu fler sätt att presentera lösningar på. Till en början har vi några enkla tumregler att hålla koll på:

\begin{itemize}
    \item En text ska vara sammanhängande och bestå av fullständiga meningar.
    \item Formler och matematiska uttryck ska vara del av fullständiga meningar.
    \item Var noga med att börja varje mening med stor bokstav och avsluta med punkt.
\end{itemize}

Utöver detta är det såklart viktigt att tänka på stavning och grammatik, precis som när vi skriver vilken annan text som helst.

%% Siffor och symboler

\subsection{Siffror och symboler}

Något som kan vara svårt i början är att hitta balansen i att kombinera siffror, symboler och ord. Det är lockande att ofta använda matematiska symboler, men en överanvändning av dessa kan i själva verket göra din text svårläslig. För att undvika detta inför vi ytterligare några tumregler:

\begin{itemize}
    \item Börja inte en mening med en siffra eller symbol om det kan undvikas.
        \begin{itemize}[leftmargin=20mm]
            \item[\textbf{Sämre:}] $\pi$ är ett rationellt tal.
            \item[\textbf{Bättre:}] Talet $\pi$ är irrationellt. 
        \end{itemize}
    \item En mening innehållandes siffror och symboler måste fortfarande vara en korrekt fullständig mening.
        \begin{itemize}[leftmargin=20mm]
            \item[\textbf{Sämre:}] $a < b \: a \neq 0$
            \item[\textbf{Bättre:}] Vi har att $a<b$ och $a \neq 0$.
            \vspace{2mm}
            \item[\textbf{Sämre:}] $x^2 - 7^2 = 0.\: x = \pm 7$
            \item[\textbf{Bättre:}] Låt $x^2 - 7^2 = 0$, då är $x = \pm 7.$
        \end{itemize}
    \item Blanda inte symboler och ord.
        \begin{itemize}[leftmargin=20mm]
            \item[\textbf{Sämre:}] Differensen $a-b$ är $<0$.
            \item[\textbf{Bättre:}] Differensen $a-b$ är negativ.
        \end{itemize}
    \item Missbruka inte implikationspilen $\Rightarrow$ eller symbolen $\therefore$ .
        \begin{itemize}[leftmargin=20mm]
            \item[\textbf{Sämre:}] $a$ är ett heltal $\Rightarrow a$ är ett rationellt tal.
            \item[\textbf{Bättre:}] Om $a$ är ett heltal så är $a$ ett rationellt tal.
            \vspace{2mm}
            \item[\textbf{Sämre:}] Vi har att $x+5=8 \therefore x = 3$
            \item[\textbf{Bättre:}]  Vi har att $x+5=8$, då följer det att $x = 3$.
        \end{itemize}
\end{itemize}




%%%%%%%%%%%%%%%%%%%%%
%%  Notation       %%
%%    - Aritmetik  %%
%%    - Mängder    %%
%%    - Olikheter  %%
%%%%%%%%%%%%%%%%%%%%%

\section{Notation}

Var försiktig och noggrann med matematisk notation. Se till att du förstår innebörden hos olika symboler innan du använder dom, till exempel betyder olika parenteser (( ), [ ], \{ \}) och pilar ($\Rightarrow$, $\Leftrightarrow$, $\to$) olika saker. Fel symbol kan göra att det du skriver inte är matematiskt korrekt, eller att du skriver någonting annat än det du faktiskt menar.


%% Aritmetik

\subsection{Aritmetik}

Notationen för de aritmetiska operationerna är välbekanta sedan grundskolan. Summa och differens av två tal $x$ och $y$ skrivs alltid som $x+y$ respektive $x-y$. Däremot finns det flera sätt att skriva produkten av $x$ och $y$:
$$xy \hspace{15mm} x\cdot y \hspace{15mm} x \times y,$$
och samma sak när det kommer till deras kvot:
$$\frac{x}{y} \hspace{15mm} x/y \hspace{15mm}  x : y.$$
Notationen $x:y$ är inte särskilt vanlig, och kan betyda olika saker i olika sammanhang, så vi undviker att använda denna. Det vanligaste är att använda $xy$ eller $x \cdot y$ för multiplikation, samt $\frac{x}{y}$ och $x/y$ för division, därför rekommenderar vi att du också gör det. Använd aldrig punkt '.' eller bokstaven 'x' för att uttrycka en produkt, detta leder direkt till förvirring hos läsaren.




%% Mängder och intervall

\subsection{Mängder och intervall}

Ett intervall är en delmängd av de rella talen av typen
$$[a,b] := \{x \in \mathbb{R}\: : \: a\leq x \leq b\}$$
där $a,b$ är rella tal och $a<b$. Det här är ett slutet intervall, vilket innebär att det innehåller ändpunkterna. Vi kan också definiera öppna intervall 
$$(a,b) := \{x \in \mathbb{R} \: : \: a<x<b\}$$
och halvöppna intervall
$$[a,b) \hspace{10mm} (a,b].$$

Notationen för det öppna intervallet krockar med notationen för ett ordnat par $(a,b) \in \mathbb{R}^2$ som man brukar använda för att till exempel uttrycka en punkt i ett koordinatsystem. Det finns alternativa sätt att uttrycka öppna och halvöppna intervall på
$$]a,b[ \hspace{10mm} [a,b[ \hspace{10mm} ]a,b],$$
men så länge det framgår tydligt från kontexten att det är ett intervall vi syftar på så fungerar båda notationerna lika bra.

Ibland vill vi uttrycka intervall som sträcker sig oändligt långt, till exempel om vi vill ange värdemängden till funktionen $f(x) = x^2$ som bekant är alla icke-negativa reella tal. Vi kan uttrycka värdemängden med mängdnotation
$$\{ a \: : \: 0 \leq a < \infty \},$$
eller med notationen för intervall
$$[0, \infty[.$$

Notera att vi använt oss av strikt olikhet och öppet intervall här då $\infty$ inte är ett tal och därför inte kan antas av funktionen.




%%%%%%%%%%%%%%%%%%%%%
%%  Struktur       %%
%%    - Inledning  %%
%%    - Huvudtext  %%
%%    - Slutsats   %%
%%%%%%%%%%%%%%%%%%%%%

\section{Struktur}

\subsection{Inledning}

\subsection{Huvudtext}

\subsection{Slutsats}



%%%%%%%%%%%%%%%%%%%%%%%%%%%
%%  Typsättning i LaTeX  %%
%%%%%%%%%%%%%%%%%%%%%%%%%%%

\section{Typsättning i \LaTeX}

Det finns inget formellt krav på att använda \LaTeX$ $ i kursen, men det är starkt rekommenderat. Att skriva \LaTeX-kod är enkelt, du omger bara det matematiska uttrycket med dollartecken. Till exempel renderas \LaTeX-koden \$1+x=5\$ som $1+x=5$. Om du vill att det matematiska uttrycket ska stå skrivet på en egen rad så omger du det istället med dubbla dollartecken, då får vi att \$\$1+x=5\$\$ renderas som $$1+x=5.$$

Fördelen med att typsätta i \LaTeX$ $ är att vi kan skriva matematik på ett snyggt och lättläst vis. Vi kan skriva alla matematiska symboler som vi vanligtvis använder med \LaTeX-kod, dessa är vanligtvis svåra att uttrycka i ett vanligt ordbehandlingsprogram. I Tabell \ref{t1} och \ref{t2} finns exempel på hur vanliga matematiska symboler och uttryck kan skrivas i \LaTeX. Exempel på lite mer komplicerade uttryck och hur dessa skrivs i \LaTeX$ $ återfinns i Tabell \ref{t3}.

\begin{table}[h]
    \begin{center}
        \begin{tabular}{| c | l |}
            \hline 
            \textbf{Rendering} & \textbf{\LaTeX-kod} \\
            \hline
            $\pi$ & \$ \textbackslash pi \$ \\
            \hline
            $\sqrt{ x }$ & \$ \textbackslash sqrt\{x\} \$ \\
            \hline
            $\equiv$ & \$ \textbackslash equiv \$ \\
            \hline
            $\cdot$ & \$ \textbackslash cdot \$ \\
            \hline
            $\mathbb{R}$ & \$ \textbackslash mathbb\{R\} \$ \\
            \hline
            $\implies$ & \$ \textbackslash implies \$ \\
            \hline
            $\pm$ & \$ \textbackslash pm \$ \\
            \hline
            $\leq$ & \$ \textbackslash leq \$ \\
            \hline
            $\geq$ & \$ \textbackslash geq \$ \\
            \hline
            $\neq$ & \$ \textbackslash neq \$ \\
            \hline
            $\infty$ & \$ \textbackslash infty \$ \\
            \hline
        \end{tabular}
        \caption{Vanliga symboler i \LaTeX.}
        \label{t1}
    \end{center}
\end{table}

\begin{table}[h]
    \begin{center}
        \begin{tabular}{| l | l |}
            \hline
            \textbf{Rendering} & \textbf{\LaTeX-kod} \\
            \hline
            $a+b$ & \$ a+b \$ \\
            \hline
            $a-b$ & \$ a-b \$ \\
            \hline
            $a \cdot b$ & \$ a \textbackslash cdot b \$ \\
            \hline
            $\frac{a}{b}$ & \$ \textbackslash frac\{a\}\{b\} \$ \\
            \hline
            $a^b$ & \$ a\textasciicircum b \$ \\
            \hline
            $f(x)$ & \$ f(x) \$ \\
            \hline
            $\sin(v)$ & \$ \textbackslash sin(v) \$ \\
            \hline
        \end{tabular}
        \caption{Vanliga uttryck i \LaTeX.}
        \label{t2}
    \end{center}
\end{table}

\begin{table}[h]
    \begin{center}
        \begin{tabular}{| l | l |}
            \hline
            \textbf{Rendering} & \textbf{\LaTeX-kod} \\
            \hline
            $5+5 \leq 10$ & \$ 5+5 \textbackslash leq 10 \$ \\
            \hline
            $2^{4\cdot3 + 1}$ & \$ 2\textasciicircum \{4 \textbackslash cdot 3 + 1\} \$ \\
            \hline
            $13 \equiv_5 3$ & \$ 13 \textbackslash equiv\_5 3 \$ \\
            \hline
            $\{a,b,c\}$ & \$ \textbackslash\{ a,b,c \textbackslash\} \$ \\
            \hline
            $f:\mathbb{Z}_{+} \to \mathbb{R}$ & \$ f: \textbackslash mathbb\{Z\}\_\{+\} \textbackslash to \textbackslash mathbb\{R\} \$ \\
            \hline
            $p(x)=x^3+5x^2-8x+2$ & \$ p(x)=x\textasciicircum 3+4x\textasciicircum 2-8x+2 \$ \\
            \hline
            $\sin(\pi)+\cos(\pi/2)$ & \$\textbackslash sin(\textbackslash pi)+\textbackslash cos(\textbackslash pi/2)\$ \\
            \hline
            $x_{1,2} = -\frac{7}{2} \pm \sqrt{5}$ & \$ x\_\{1,2\} = -\textbackslash frac\{p\}\{2\} \textbackslash pm \textbackslash sqrt\{5\} \$ \\
            \hline
        \end{tabular}
        \caption{\LaTeX-exempel.}
        \label{t3}
    \end{center}
\end{table}


%%%%%%%%%%%%%%%
%%  Exempel  %%
%%%%%%%%%%%%%%%

\section{Exempel}



% \begin{center}
%     \noindent\fbox{
%         \parbox{0.8\textwidth}{
%             \textbf{Problem: }Motivera att $(a^m)^n = a^{mn}$.
%         }
%     }
%     \end{center}
    
%     \begin{center}
%         \noindent\fbox{
%             \parbox{0.8\textwidth}{
%                 \textbf{Mindre bra lösning:}
    
%                 Vi kan motivera detta genom att visa varför (a\textasciicircum 2)\textasciicircum 3=a\textasciicircum 6:
%                 (a\textasciicircum 2)\textasciicircum 3=(a*a)\textasciicircum 3=(a*a)*(a*a)*(a*a)=a*a*a*a*a*a=a\textasciicircum 6.
%             }
%         }
%     \end{center}
    
%     \begin{center}
%         \noindent\fbox{
%             \parbox{0.8\textwidth}{
%                 \textbf{Bättre lösning:}
    
%                 Vi kan motivera detta genom att visa varför $(a^2)^3 = a^6$: \newline
%                 $(a^2)^3 = (a \cdot a)^3= (a \cdot a)\cdot (a \cdot a) \cdot (a \cdot a) = a \cdot a \cdot a \cdot a \cdot a \cdot a = a^6$.
%             }
%         }
%     \end{center}


% \begin{center}
%     \noindent\fbox{
%         \parbox{0.8\textwidth}{
%             \textbf{Problem: }Faktorisera polynomet $p(x)=x^2-5x+6$.
%         }
%     }
% \end{center}
% \begin{center}
%     \noindent\fbox{
%         \parbox{0.8\textwidth}{
%             \textbf{Mindre bra lösning:}

%             $x^2-5x+6 = 0 => x=5/2 \pm$ rotenur$1/4$

%             Rötterna är $x_1 = 2$, $x_2 = 3$. Det ger faktoriseringen $(x-2)(x-3)$.
%         }
%     }
% \end{center}
% \begin{center}
%     \noindent\fbox{
%         \parbox{0.8\textwidth}{
%             \textbf{Bättre lösning:}

%             Vi börjar med att sätta $p(x)=0$ och löser ekvationen genom att kvadratkomplettera:
%             \vspace{2mm}

%             $\quad \: x^2-5x+6 = 0$

%             $\Leftrightarrow x^2-5x = -6$

%             $\Leftrightarrow(x-\frac{5}{2})^2 = \frac{1}{4}$
%             \vspace{2mm}

%             Nu drar vi roten ur båda led:
%             \vspace{2mm}

%             $\quad \: x-\frac{5}{2} = \pm \sqrt{\frac{1}{4}}$

%             $\Leftrightarrow x = \frac{5}{2} \pm \frac{1}{2}$
%             \vspace{2mm}

%             Detta ger oss lösningarna $x_1 = 2$ och $x_2=3$.

%             Enligt faktorsatsen är $(x-a)$ en faktor till polynomet $p(x)$ om och endast om $x = a$ är en rot till $p(x)$. Det följer av satsen, och att ett andragradspolynom inte kan ha fler än två rötter, att $p(x) = (x-2)(x-3)$.
%         }
%     }
% \end{center}

% \begin{center}
%     \noindent\fbox{
%         \parbox{0.8\textwidth}{

%         }
%     }
% \end{center}



% \begin{center}
% \begin{tcolorbox}[width=\linewidth,colback={red!30!white},title={\textbf{Mindre bra lösning}},outer arc=0mm,colupper=black]

% \end{tcolorbox} 
% \end{center}


\begin{center}
\begin{tcolorbox}[width=\linewidth,colback={white},title={\textbf{Exempel 1}},outer arc=0mm,colupper=black]
    Motivera att $(a^m)^n = a^{mn}$.
\end{tcolorbox} 
\end{center}

\begin{center}
\begin{tcolorbox}[width=\linewidth,colback={red!25!white},title={\textbf{Exempel 1: Lösning - Sämre}},outer arc=0mm,colupper=black]
    Vi kan motivera detta genom att visa varför (a\textasciicircum 2)\textasciicircum 3=a\textasciicircum 6:
    (a\textasciicircum 2)\textasciicircum 3=(a*a)\textasciicircum 3=(a*a)*(a*a)*(a*a)=a*a*a*a*a*a=a\textasciicircum 6.
\end{tcolorbox} 
\end{center}

\begin{center}
\begin{tcolorbox}[width=\linewidth,colback={green!25!white},title={\textbf{Exempel 1: Lösning - Bättre}},outer arc=0mm,colupper=black]
    Vi kan motivera detta genom att visa varför $(a^2)^3 = a^6$: \newline
    $(a^2)^3 = (a \cdot a)^3= (a \cdot a)\cdot (a \cdot a) \cdot (a \cdot a) = a \cdot a \cdot a \cdot a \cdot a \cdot a = a^6$.
\end{tcolorbox} 
\end{center}







\begin{center}
\begin{tcolorbox}[width=\linewidth,colback={white},title={\textbf{Exempel 2}},outer arc=0mm,colupper=black]
    Faktorisera polynomet $p(x)=x^2-5x+6$.
\end{tcolorbox} 
\end{center}

\begin{center}
\begin{tcolorbox}[width=\linewidth,colback={red!25!white},title={\textbf{Exempel 2: Lösning - Sämre}},outer arc=0mm,colupper=black]
    $x^2-5x+6 = 0 => x=5/2 \pm$ rotenur$1/4$

    Rötterna är $x_1 = 2$, $x_2 = 3$. Det ger faktoriseringen $(x-2)(x-3)$.
\end{tcolorbox} 
\end{center}



\begin{center}
\begin{tcolorbox}[width=\linewidth,colback={green!25!white},title={\textbf{Exempel 2: Lösning - Bättre}},outer arc=0mm,colupper=black]    
    Vi börjar med att sätta $p(x)=0$ och löser ekvationen genom att kvadratkomplettera:
    \vspace{2mm}

    $\quad \: x^2-5x+6 = 0$

    $\Leftrightarrow x^2-5x = -6$

    $\Leftrightarrow(x-\frac{5}{2})^2 = \frac{1}{4}$
    \vspace{2mm}

    Nu drar vi roten ur båda led:
    \vspace{2mm}

    $\quad \: x-\frac{5}{2} = \pm \sqrt{\frac{1}{4}}$

    $\Leftrightarrow x = \frac{5}{2} \pm \frac{1}{2}$
    \vspace{2mm}

    Detta ger oss lösningarna $x_1 = 2$ och $x_2=3$.

    Enligt faktorsatsen är $(x-a)$ en faktor till polynomet $p(x)$ om och endast om $x = a$ är en rot till $p(x)$. Det följer av satsen, och att ett andragradspolynom inte kan ha fler än två rötter, att $p(x) = (x-2)(x-3)$.
\end{tcolorbox} 
\end{center}






\begin{center}
\begin{tcolorbox}[width=\linewidth,colback={white},title={\textbf{Exempel 3}},outer arc=0mm,colupper=black]

\end{tcolorbox} 
\end{center}

\begin{center}
\begin{tcolorbox}[width=\linewidth,colback={red!25!white},title={\textbf{Exempel 3: Lösning - Sämre}},outer arc=0mm,colupper=black]

\end{tcolorbox} 
\end{center}

\begin{center}
\begin{tcolorbox}[width=\linewidth,colback={green!25!white},title={\textbf{Exempel 3: Lösning - Bättre}},outer arc=0mm,colupper=black]

\end{tcolorbox} 
\end{center}






%%%%%%%%%%%%%%%%
%%  Övningar  %%
%%%%%%%%%%%%%%%%

\section{Övningar}
\begin{enumerate}
    \item Skriv om följande meningar så att de följer tumreglerna för hur man skriver matematisk text:
    \begin{enumerate}[label=(\alph*)]
        \item $x$ är negativ $\therefore \sqrt{x}$ är ett komplext tal.
        \item $x^2 = 4 \Rightarrow x=2 \vee x=-2$.
        \item $a$ är $>0$.
        \item Om $x>0$ $g(x) \neq 0$.
        \item $\sin(\pi x)= 0\Rightarrow x$ är ett heltal. 
    \end{enumerate}
    \item Beskriv följande mängder med ord:
    \begin{enumerate}
        \item $\{x \in \mathbb{Q} \: : \: 0<x<1\}$
        \item $\{x \in \mathbb{Z} \: : \: 0 < x \wedge 2 \mid x\}$
    \end{enumerate}
    \item Förklara utan att använda några matematiska symboler:
    \begin{enumerate}[label=(\alph*)]
        \item Hur dividerar man två bråk med varandra?
        \item Hur utför man kvadratkomplettering?
        \item Givet ett positivt heltal $x$, hur tar man reda på om $x$ är ett primtal?
        \item Givet ett positivt heltal $x$, hur tar man reda på om $x$ är summan av två kvadrater?
    \end{enumerate}
\end{enumerate}


% References
\newpage
\bibliography{refs}


\end{document}